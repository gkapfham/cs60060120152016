%!TEX root = ./cs600Fall2015_syllabus.tex

\input syllabuspre
\begin{document}
\MYTITLE{Syllabus}
\MYHEADERS{Syllabus}{}

\vspace*{-.3in}
\subsection*{Course Instructors}

\begin{tabular}{c c}

\begin{minipage}{3.5in}
Janyl Jumadinova\\
\noindent Office Location: Alden Hall 105 \\
\noindent Email: \url{jjumadinova@allegheny.edu} \\
\end{minipage} &

\begin{minipage}{3.5in}
Gregory M.\ Kapfhammer\\
\noindent Office Location: Alden Hall 108 \\
\noindent Email: \url{gkapfham@allegheny.edu} \\
\end{minipage} \\

\begin{minipage}{3.5in}
Robert S.\ Roos\\
\noindent Office Location: Alden Hall 106 \\
\noindent Email: \url{rroos@allegheny.edu} \\
\end{minipage} &

\begin{minipage}{3.5in}
John Wenskovitch\\
\noindent Office Location: Alden Hall 104 \\
\noindent Email: \url{jwenskovitch@allegheny.edu} \\
\end{minipage}

\end{tabular}
\vspace*{-.3in}

\subsection*{Instructors' Office Hours}

Please visit the Web sites of the course instructors to view their office hours.  Using the ``appointment slots''
feature of Google Calendar, you can select an available meeting time. After picking your time slot, the reserved meeting
will appear in both your Google Calendar and the instructor's.

\vspace*{-.1in}
\begin{itemize}
    \itemsep -.25em
        \item Janyl Jumadinova: \url{http://www.cs.allegheny.edu/sites/jjumadinova/}
        \item Gregory M.\ Kapfhammer: \url{http://www.cs.allegheny.edu/sites/gkapfham/}
        \item Robert S.\ Roos: \url{http://www.cs.allegheny.edu/sites/rroos/}
        \item John A. Wenskovitch: \url{http://www.cs.allegheny.edu/sites/jwenskovitch/}
\end{itemize}

\vspace*{-.25in}
\subsection*{Course Communication}

Throughout the semester, students and faculty will use Slack to support course communication. All students will be
required to integrate notifications from the version control repositories with a specified Slack channel, thereby
allowing all students and the course instructors to observe everyone's progress on their thesis research.  Whenever
possible, students are also encouraged to post appropriate questions to a channel in Slack, which is available at
\url{https://CMPSC600Fall2015.slack.com}.

\vspace*{-.1in}
\subsection*{Course Schedule}

% {\bf CMPSC 600}
\begin{center}
\begin{tabular}{r|l}

\hline

August 24                & Submit request for first and second reader \\
September 7              & Register for CMPSC 600 with first reader \\
November 2 -- November 8 & Register for CMPSC 610 with first reader \\
October 20 -- December 1 & Oral defense of thesis proposal \\
December 15              & Submit two chapters to course instructor by 5 pm \\

\hline

Before September 1 & Schedule weekly meeting time with your first reader \\
Before September 8 & Create version control repositories for your research \\
Before September 15 & Integrate version control repositories with Slack \\
Before October 6   & Schedule proposal defense with Pauline Lanzine \\
Before November 3  & Secure formal approval of proposal from first reader \\
Before November 24 & Get technical report number from Pauline Lanzine \\

\hline

Entire academic semester   & Tuesday class session, 1:30 pm -- 2:20 pm \\
September through December & Meet with first reader on a regular basis \\
September through December & Communicate with instructors and students in Slack \\

\hline

\end{tabular}
\end{center}

\noindent The schedule for CMPSC 600 may change as the course instructors deem appropriate. Please note that, unless
evidence of extenuating circumstances is presented in writing to all of the course instructors, a student's grade in the
course will be reduced if the stated deadlines are not met. Students who have questions or concerns about these
deadlines should talk with their first reader.

\vspace{-.15in}
\subsection*{Required Textbooks}
\vspace{-.05in}

\noindent{\em On Being a Scientist: A Guide to Responsible Conduct in Research} (Third Edition).  Committee on Science,
Engineering, and Public Policy, National Academy of Sciences, National Academy of Engineering, and Institute of
Medicine. ISBN: 0309119715, 82 pages, 2009.\\ (References to the textbook are abbreviated as ``OBAS'').

\noindent{\em BUGS in Writing: A Guide to Debugging Your Prose} (Second Edition). Lyn Dupr\'e.  Addison-Wesley
Professional.  ISBN-10: 020137921X and ISBN-13: 978-0201379211, 704 pages, 1998.\\ (References to the textbook are
abbreviated as ``BIW'').

\noindent{\em Writing for Computer Science} (Second Edition).  Justin Zobel.  Springer ISBN-10: 1852338024 and ISBN-13:
978-1852338022, 270 pages, 2004. \\ (References to the textbook are abbreviated as ``WFCS'').

\vspace*{-.15in}
\subsection*{Overview of the Grading Policies}

Final grades are determined after the entire faculty of the Department of Computer Science, not just the course
instructor for CMPSC 600, review and discuss all of the submitted deliverables.

Your grade in CMPSC 600 will be based on a combination of the following activities and deliverables. Percentages are not
given because we recognize that the senior thesis experience differs from one student to the next and that there are many
variables, such as the nature of the project and the availability of external resources, that can influence the relative
importance of these criteria. However, it is important to note that a large percentage of your grade depends upon your
written thesis proposal, the oral defense of your thesis proposal, and your two chapters.

\vspace*{-.05in}

\begin{itemize}
  \itemsep -.25em

  \item {\bf Class Participation}: This includes meeting regularly with your first reader. Although the exact details
    about frequency and length of each meeting must be established with your first reader, you should adhere to the
    previously stated schedule. Additionally, this also involves regular contributions, in the form of questions and
    comments, to the course's Slack team.

  \item {\bf Course Repositories}: This involves students creating, at minimum, two version control repositories to
    store (i) their thesis proposal and written chapters and (ii) any relevant source code and data.  Both of your
    readers must have administrative access to your repositories.

    % Additionally, you repositories must be correctly integrated into the appropriate channel in our Slack team. All
    % repositories must have a README.md file that contains full-featured and well-written instructions for creating all
    % of the deliverables under version control.

  \item {\bf Written Proposal}: This document must be approved, in writing by the specified deadline, by the first
    reader for your senior thesis and formatted according to the department's thesis proposal style requirements,
    which is available from the course Web site.

  \item {\bf Proposal Defense}: This event is scheduled in consultation with your first and second reader and the
    building coordinator, Pauline Lanzine. Students may not schedule their proposal defense until their thesis proposal
    has been formally approved, in writing, by their first reader; evidence of this approval must be submitted to
    Pauline Lanzine when scheduling the defense.

  \item {\bf Thesis Chapters}: Any two chapters of your final senior thesis must be submitted to the course instructor
    by the aforementioned deadline.  Written in a professional and scientific style, these chapters must be formatted in
    the department's thesis style; please note that this style is available from the course Web site and it is different
    from the proposal's style.

\end{itemize}

\vspace*{-.25in}
\subsection*{Details About Course Expectations and Deliverables}

\noindent{\bf Class Participation}: Once your readers have been assigned, you must regularly meet with your first
reader, who will report on your participation when the department's faculty meet to assign final grades.  Students are
expected to come to each meeting with a status update on their progress and a meeting agenda.  Students should conclude
each meeting by listing the tasks that they want to complete before the next meeting. Evidence of regular meetings must
be submitted to the course instructor. In addition, students should regularly participate in the discussions on the
relevant channels in the Slack team for our course. Your participation on Slack may involve giving a quick status update
to your first reader, inviting your first reader to examine a draft of your proposal or compile and run a new version of
a program, or, within the bounds of the Honor Code, answering a question from another senior conducting thesis research.
Finally, all students are required to attend and actively participate in all of the class sessions that will involve
both meetings with their research group and informal discussion during the departmental coffee and tea session.

\noindent{\bf Course Repositories}: Every student must create at least two version control repositories to store (i)
their thesis proposal and written chapters and (ii) any relevant source code and data; students may create additional
repositories in consultation with their first reader. Unless advised by their first reader to do otherwise, students are
expected to create their repositories in Bitbucket. Regardless of the repository provided that you select, your first
and second reader must have administrative access to your repositories. Additionally, your repositories must be
correctly integrated into the appropriate channel in our Slack team, thereby allowing all faculty and students to see
everyone's progress on their research. Unless extenuating circumstances prevent you from doing so, you should commit to
all of your repositories every week. Finally, all repositories must have a README.md file that contains well-written
instructions for creating all of the deliverables under version control.

\noindent{\bf Thesis Proposal}: The proposal should follow the department's proposal style and thus must include an
abstract, the main body of your proposal, a tentative schedule for completing the project, a bibliography, and any other
information deemed important by your first reader. This will often include one or more of the following: a survey of the
existing literature; an overview of your proposed technique; technical diagrams and formal statements of algorithms
illustrating your main approach; the description of an evaluation method; examples or code artifacts or other evidence that
you understand the nature of the work you are proposing and can feasibly complete it in the time available.  Finally, the
proposal must fully adhere to professional standards of writing.

Although your first reader will be your primary contact person as your write and revise your thesis proposal, you may
involve your second reader as appropriate. Primarily, your first reader will make suggestions on your submitted drafts;
students are expected to revise multiple proposal drafts.  You must work at a pace that will ensure that your first
reader can formally approve the final draft of your thesis proposal before the stated deadline.  Failure to secure
formal approval of your proposal before this date will result in the reduction of your final grade in CMPSC 600.
Securing formal approval of your thesis involves a student having their first reader sign and date the final printed
version of the thesis proposal. This document must be shown to Pauline Lanzine when scheduling your proposal defense; no
defense will be scheduled without this evidence of approval.

\noindent{\bf Proposal Defense}: A proposal defense is a prepared, formal presentation of about ten minutes in which you
lay out the essential parts of your chosen project under the assumption that your first and second reader have studied
your proposal.  Following the presentation that is supported by polished slides, you will participate in a discussion
with your readers to identify potential challenges, refine or modify some aspects of the project proposal, and ensure that
your project is feasible and appropriate. All aspects of your proposal defense should be prepared in consultation with
your first reader.  You must schedule your proposal defense before the stated deadline. Your grade in CMPSC 600 will be
reduced if you miss the deadline for scheduling or conducting your defense.

\noindent{\bf Thesis Chapters}: Your two chapters, due on the previously stated date, should represent a significant
addition to or extension of the material in your proposal. Don't simply ``split the proposal into two chapters'' ---
this usually does not work well since your chapters represent work completed, not work being proposed.  Chapters are
judged according to the same professional standards as the proposal; they must include a full bibliography, a
preliminary table of contents, lists of any figures and tables, and any other items required by your first reader.

As you write your chapters in consultation with your first and second reader, allow these individuals to comment on your
drafts and then make all of their requested changes.  Plan to write several drafts of the chapters before submitting them
on the due date; failure to submit the chapters by the stated deadline will result in the reduction of your final grade
in CMPSC 600.

\subsubsection*{Seeking Assistance}

Students who are struggling to understand the knowledge and skills developed in a class or laboratory session are
encourage to seek assistance from the course instructor. Throughout the semester, students should, within the bounds of
the Honor Code, ask and answer questions on the Slack site for our course; please request assistance from the instructor
first through Slack before sending an email. Students who need the course instructor's assistance must schedule a
meeting through his Web site and come to the meeting with all of the details needed to discuss their question.

\vspace*{-.15in}
\subsubsection*{Using Email}
\vspace*{-.05in}

Although we will primarily use Slack for class communication, I will sometimes use email to send announcements about
important matters such as changes in the schedule. It is your responsibility to check your email at least once a day and to
ensure that you can reliably send and receive emails. This class policy is based on the statement about the use of email that
appears in {\em The Compass}, the College's student handbook; please see the instructor if you do not have this
handbook.

\subsection*{Disability Services}

The Americans with Disabilities Act (ADA) is a federal anti-discrimination statute that provides comprehensive civil
rights protection for persons with disabilities.  Among other things, this legislation requires all students with
disabilities be guaranteed a learning environment that provides for reasonable accommodation of their disabilities.
Students with disabilities who believe they may need accommodations in this class are encouraged to contact Disability
Services at 332-2898.  Disability Services is part of the Learning Commons and is located in Pelletier Library.
Please do this as soon as possible to ensure that approved accommodations are implemented in a timely fashion.

\subsection*{Honor Code}

The Academic Honor Program that governs the academic program at Allegheny College is described in the Allegheny
Course Catalogue.  The Honor Program applies to all work that is submitted for academic credit or to meet non-credit
requirements for graduation at Allegheny College.  This includes all work assigned for these classes (e.g., source code,
technical diagrams, and your written content); deliverables that are nearly identical the work of others will be taken
as evidence of violating the Honor Code. All students who have enrolled in the College will work under the Honor
Program.  Each student who has matriculated at the College has acknowledged the following pledge:

\vspace*{-.1in}
\begin{quote}
I hereby recognize and pledge to fulfill my responsibilities, as defined in the Honor Code, and to maintain the
integrity of both myself and the College community as a whole.
\end{quote}
\vspace*{-.15in}

\subsection*{Welcome to an Adventure in Computer Science}

CMPSC 600 affords you the opportunity to pursue independent research in computer science.  Moreover, these
courses properly position you to conduct ground-breaking work that can have a positive influence on your future career
and graduate school prospects, the students and faculty in the Department of Computer Science, the Allegheny College
community, and a broader society that heavily relies on computer hardware and software.  At the start of your senior
year, I invite you to pursue this class with great enthusiasm and vigor.

% \subsection*{Special Needs and Disabilities}
% The Americans with Disabilities Act (ADA) is a federal anti-discrimination
% statute that provides comprehensive civil rights protection for persons
% with disabilities. Among other things, this legislation requires that
% all students with disabilities be guaranteed a learning environment
% that provides for reasonable accommodation of their disabilities.
% If you believe  you have a disability requiring an accommodation,
% please contact the Learning Commons at 332-2898.
%
\end{document}
